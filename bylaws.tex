\documentclass[12pt,letterpaper]{article}
\title{BYLAWS} 
\author{Pittsburgh Association of the Deaf, Inc.}
\date{July 2023}

\usepackage{draftwatermark}
\SetWatermarkScale{5}

\begin{document}
\maketitle
\newpage
\tableofcontents
\newpage

\section{Name}

The name of this organization is the Pittsburgh Association of the Deaf, Incorporated. 

\section{Purpose}
The purpose of this Association is to maintain a club for the education and social enjoyment of its members. 

\section{Location}
The headquarters of this Association will be in Allegheny County at 1854 Forbes Avenue, Pittsburgh, Pennsylvania.

\section{Members}

\subsection{Classes of Membership}
There will be six classes of members: Active, Associate, Senior Citizen, College Student, Life and Social. 

\subsubsection{Active Members}
Active members will enjoy all the privileges of the club.

There will be a limit of 90 Active members. When there are less than 90 Active members, the Association may admit Associate members to Active status by 2/3 vote at the next meeting.

Associate members with two years membership and requesting transfer to Active status will have first privileges.

No hearing member can become an Active member unless they are the spouse of a deaf Active member, or the domestic (household) of a deaf Active member. 

\subsubsection{Associate Members}
Associate members will have all of the privileges of Active members except they cannot vote or hold office. There is no limit on the number of Associate members. 

\subsubsection{Senior Citizen Members}
Senior Citizen membership is open to members who:
\begin{itemize}
\item Are 65 years of age or older 
\item or are widows 62 years of age or older
\item Have been an Associate member for at least one year. 
\end{itemize}

They will enjoy the same privileges as Associate members. There is no limit on the number of Senior Citizen members.

\subsubsection{Life Member}
Life membership is open to members who:
\begin{itemize}
\item Are 65 years of age, or retired from employment, or becomes disabled
\item Have been an Active member for 20 consecutive years or more
\end{itemize}
Life members will retain their rights as Active members except they cannot hold office. 

\subsubsection{College Student}
College Student membership is open to any persons who are still in college. They must provide college student identification. %Need to add what privileges college members have

\subsubsection{Social Members}
Social members privileges are limited to entering the club, consuming alcoholic beverages, and participating in social activities. Membership is valid only for the day they apply. Social Members cannot upgrade their membership to other classes, they will have to re-apply.

\subsection{Application}

\subsubsection{Associate and College Students}
Candidates for Associate and College Student membership must apply with the consent of two Active or Associate members. The initiation fee and appropriate dues must be collected with the application. They must be investigated by a Membership Committee and voted on at the next meeting.

\subsubsection{Active Membership}
{\textless}reserved{\textgreater}

\subsubsection{Social Membership}
Candidates for Social membership must apply with the consent of two Active or Associate members. Dues must be collected immediately with the application.

\subsection{Fees and Dues}

\subsubsection{Initiation Fees}
The initiation fee for Associate member and College Student will be \$5, paid with the application.

\subsubsection{Dues}
All dues, except for Social members, are due by March 31 every year. 

The annual dues are:
\begin{itemize}
\item Active Member -- \$45
\item Associate Member -- \$35
\item First year Associate Member -- \$20. This discount applies only for someone who has never been a member or was not
a member for five years or longer.
\item Senior Citizen Member -- \$20
\item Life Member -- \$15
\item College Student Members -- \$15
\item Social Members -- \$5 per visit submitted with application.
\end{itemize}

A late penalty fee of \$5 will be assessed for dues not paid by the deadline.

A member who is late with payment for four months will be suspended and cannot enjoy any of the club privileges until they pay.

%Need to clarify the conflicts here.
Any member who is late with payment for six months will be dropped from the roll unless satisfactory reasons are given within two years.

A dropped member, being readmitted to membership, must pay up their back dues, penalty fees, and any other obligations.

If by the sixth month the member is still in arrears, they will be dropped from the roll. They will then have to apply for membership as a new member and pay the regular initiation fee to be reinstated. There must be no exception to this rule. This rule governs all classes of members of the Association. 

\subsection{Duties and Conduct of Members}
It will be the duty of members to:
\begin{itemize}
\item Observe order during the meetings
\item Cooperate with the officers of the Association in enforcing Bylaws and rules
\item Inform a non-member that only members can be admitted to the premises
\item Refrain from all manners of noise, disputes, personalities
\item Refrain from any act or conduct injurious to the order, peace, interest and welfare of the Association.
\item Keep the Secretary duly informed of their change of address. 
\end{itemize}

\subsection{Discipline}

\subsubsection{Suspension}
Any member who has been suspended from membership by action of the Board of Directors, or by vote of the Association as a whole, must be barred from all affairs of the Association and cannot enter the club's premises for the duration of suspension.

\subsubsection{Expulsion}
Any member who has been expelled must not be readmitted to membership for a minimum of 2 years from the date of expulsion. A longer period may be fixed.

\subsubsection{Authority}
The Board of Directors may also take such action as it sees fit on a complaint made by one members against another, if the ground of the complaint is properly an Association affair -- that is to say, the Board of Directors will have full power to act immediately and forthwith on such cases and settle them out and amends, may impose a reasonable fine to cover damage done or as a penalty, and may even suspend such a member for a fixed or indefinite period during which they should be barred from the premises of the Association and from attendance at its affairs. Or, the Board of Directors may take the case under advisement or investigation, consideration and later action. Or, it may refer the matter to a meeting of the Association as a whole. If the charges against said member merit expulsion, they may be expelled by a 2/3 vote of the qualified members present at the meeting.

\subsubsection{Appeal}
A member disciplined by the Board of Directors may appeal within 30 days of the decision. If the appeal request is accpeted by a majority of the members present at a meeting, the member will be given the opportunity to state their side of the matter at issues.

\subsubsection{Resignation}
A member wishing to quit must submit their resignation in writing. If all their dues and other obligations are paid in full, the resignation may be accepted by a majority vote at any meeting without hurting their application for reinstatement in the future.

\section{Officers}

\subsection{Offices}
The officers of this Association must consist of a President, a First Vice President, a Second Vice-President, a Third Vice-President, a Secretary, a Treasurer, an Assistant Treasurer, a Membership Secretary, and three Trustees. 

\subsection{Eligibility}
To be eligible to hold office, an Active member must:
\begin{itemize}
\item Have been an active member for at least 1 year.
\item Be in good financial standing.
\item Have attended at least six regular meetings during odd numbered year.
\item Have attended at least six regular meetings during even numbered year before the election.
\end{itemize}

\subsection{Election}
The officers will be elected by ballot at the regular meeting held in December during an even-numbered year. 

Active members, with the attendance of less than four regular meetings during odd numbered year and less than four regular meetings before the election during even numbered year, cannot vote for new officers. 

\subsection{Terms of Office}
The newly elected officers will be sworn in immediately after their election and must take office on January 1 of the following odd numbered year for a term of 2 years and until their successors are elected. 

If any vacancy occurs in any office other than the President, the President will have the power to appoint an Active member to serve until the next meeting, when the Association must vote to fill the vacancy.

\subsection{Duties of Officers}
All officers will perform the duties prescribed by these bylaws and by the parliamentary authority adopted by the Association.

\subsubsection{President}
The President must appoint members to the standing committees for the year at the first meeting in January. 

\subsubsection{Vice Presidents}
If the President is absent or disabled, the powers and duties must transfer to the highest ranking Vice President available.

\subsubsection{Secretary}
The Secretary must inform the membership of a motion that has been carried but not acted upon within 60 days.

\subsubsection{Treasurer}
The Treasurer must deposit all monies in a bank designated by the Board of Directors. Payments of bills against the Association must be signed by both the President and the Treasurer. The Treasurer will assist the Trustees with the development of the club budget. 

\subsubsection{Assistant Treasurer}
The Assistant Treasurer will help the Treasurer with their duties.

\subsubsection{Membership Secretary}
The Membership Secretary will collect the monthly dues of the members, keep a record of such payments, and turn all monies collected over to the Treasurer before the monthly audit by the Trustees.

\subsubsection{Trustees}
The Trustees must audit all finances monthly and report the results at the monthly meeting.

The trustees may have responsibilities which include but are not limited to:
\begin{itemize}
\item Developing and overseeing the budget process
\item Providing for financial controls
\item Analysis of financial reports
\item Review and approval of annual operations
\item Review of extraordinary expenditures exceeding \$200
\item Review of investment policies for liquid assets of the organization.
\end{itemize}

Trustees cannot serve on any committees.

\subsection{Bond}
The Treasurer, Membership Secretary and Club Manager must be bonded by a surety company for an amount to be fixed by the Board of Directors.

\subsection{Resignation}
An Officer must resign in writing with a valid reason. A majority vote at the next regular meeting will be needed to accept the resignation.

\section{Meetings}

\subsection{Regular Meeting}

\subsubsection{Date and Time}
The club's regular business meeting will be held monthly on the second Saturday at 7:30 p.m.\ unless some other date or time has been set.

\subsubsection{Quorum}
20 Active members constitutes a quorum at any meeting. Members may be excused from the meeting at the President's discretion, but only when the quorum is not affected. 

\subsubsection{Order of Business}
The regular order of business will be: 
\begin{enumerate}
\item Salutation to the United States Flag 
\item Necrology 
\item Roll Call 
\item Reading of the Minutes of last meeting 
\item Communications 
\item Report of Officers 
\item Report of Committees 
\item Application for Membership 
\item Admission of New Members 
\item Election and Installation of New Officers 
\item Unfinished Business 
\item New Business 
\item Adjournment 
\end{enumerate}

This order may be suspended by the Association at any meeting. 

\subsubsection{Limit on Spending}
Any motion passed in a meeting requiring \$200 or more from the General Fund, is subject to review and approval by the Board of Directors. A majority of the entire board is required for approval. If the board rejects the motion, the motion is sent back to the Active members. If the same motion is passed by the Active members a second time, it will take effect immediately.

\subsection{Special Meeting}
A special meeting may be called by order of the President, or at the request of five or more members in good standing.

\section{Board of Directors}

\subsection{Board Members}
The officers will constitute the Board of Directors.

\subsection{Board Powers}
The Board of Directors will have the full power to deal accordingly on the spot in any manner that they see fit with members guilty of conduct unbecoming a gentle person, or in violation of Section 1 of this Article on the Association's premises or at dances, picnics or other affairs held under the auspices of the Association. 

The Board of Directors will have charge of the property and effects of the Association. They will have charge of the rooms and see they are kept in good order. They will receive and take charge of all gifts, of books, pictures, etc. They must make, or direct to be made, all purchases ordered by the Association. They must order the payment of all necessary expenses and transact all business duties not otherwise herein ordered.

\section{Club Manager}

\subsection{Election}
The Club Manager will be elected by ballot at the regular meeting held in December during an odd-numbered year. The manager will assume responsibility on January 1 of the following even numbered year for a term of 2 years.

The Club Manager must be an Active member before they are appointed. 

\subsection{Duties}
The Club Manager duties will be to keep the club clean, to maintain order at all times, to see that only members are admitted to the clubrooms. They will account to the Treasurer for all monies taken at the bar and all other sources of income in the clubrooms. The Club Manager will have the privileges of appointing assistants.

\subsection{PLCB Rules}
The club is governed by the rules and regulations of the Pennsylvania Liquor Control Board.

The Club Manager must purchase all beer, wine, and whiskeys and keep records as required by the Pennsylvania Liquor Control Board.

\subsection{Oversight}
The Club Manager and his/her assistants will be under the jurisdiction of the Board of Directors. 

\section{Standing Committees}
The standing committees of the Association will be the Law, Membership, House and Bingo Night Committees. The President will appoint the chair and members of each standing committee. 

\subsection{Law}
The Law Committee will consist of five members. The committee must receive proposed amendments to the Bylaws that are referred from the regular meeting. They will review and discuss the proposals, and make recommendations to the Association at the next regular meeting.

\subsection{Membership}
The Membership Committee will consist of several Active Members. The committee will oversee all applications for membership in the Association. They must investigate all applicants before being submitted to the Association at the regular meeting. Also, the committee will be responsible for checking people's membership status when they enter the premises.

\subsection{House}
The House Committee consists of three members. The committee will be responsible for the maintenance and repairs of the building's interior and exterior as needed or as ordered by the President or the Club Manager. The cost of repair or purchases must be approved by the President.

\subsection{Game Event}
The Game Event Committees will consist of a certain number of members as appointed by the chairperson of each Game Event Committees as instructed by the policy and/or guideline. Game Event Committees will be responsible for the donation at the door, games, and prizes during the game event. They must account to the Treasurer for all monies taken from the donation and games, along with the list of distributed prizes. 

\section{Parliamentary Authority}
Meetings will be governed by Robert's Rules of Order, except when the rules conflict with these Bylaws; and, provided, that a point of order and question of procedures referred by the chair to a general vote must be settled by a majority vote of the members present.

\section{Amendments to the Bylaws}

\subsection{Proposals}
All proposed amendments to the Bylaws made at a regular business meeting must be in writing and will be referred without debate to the Law Committee.

\subsection{Review and Recommendation}
The Law Committee must consider all proposed amendments or alterations in the Bylaws and will recommend their adoption or rejection at a regular meeting. 

\subsection{Adoption}
A 2/3 vote of the members present is necessary to adopt any amendment to the Bylaws.

\section{Dissolution}

\subsection{Liquidation}
In the event of dissolution of this organization, all club assets remaining after all legal obligations have been paid and the building has been sold, must be placed in a fund in a local bank under the trusteeship of three active members chosen by the membership; such funds to remain until the club reorganizes. 

\subsection{Disbursement}
However, at the expiration of the 10th year after the dissolution, the fund, at the discretion of the trustees, must be given to other organizations of good standing which promote the welfare of the Deaf. 

\end{document}
